\section{レンズの歪曲収差とは何か}
収差を大別すると、単色収差と色収差に分けられる。歪曲収差は単色収差のひとつである。
単色収差には記述方法の異なる光線収差と波面収差の分類があるが、物理的実体は同じであり、
光線収差を求めるために、屈折面においてスネルの法則を用いて光線を追跡するとき、近似していくなかで、光線収差には5つの分類が出来上がる。これはザイデルの5収差と呼ばれ、歪曲収差はそのひとつとされている。
歪曲収差は光軸に垂直な平面上にある物体は、光軸に垂直な平面上に結像するため、像は鮮明となる。しかし、光軸から離れた部分ほど形状が歪む性質が観測できる。歪む形には糸巻き型や樽型が挙げられる。
歪曲収差は像点のずれが像高の3乗に比例するため、像の大きさによって横倍率が異なることにより生じる。つまり、入射光線の画角または像高により、結像倍率が異なってしまう。

\begin{figure}[h]
	\centering
	\includegraphics[width=50mm]{image/dummy.png.eps}
	\caption{歪曲収差}
	\label{caption1}
\end{figure}

\begin{figure}[h]
    
    \centering
    \subfloat[糸巻き型]{\includegraphics[width=50mm]{image/dummy.png.eps} \label{shape1}}
    \subfloat[樽型]{\includegraphics[width=50mm]{image/dummy.png.eps} \label{shape2}}
    
    \caption{歪曲収差による観測できる形}
    \label{shape}
    \end{figure}