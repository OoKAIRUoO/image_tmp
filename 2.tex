\section{レンズの歪曲収差の校正方法について}
歪曲収差は半径方向歪みに分類され、これはレンズの形状に起因する。
レンズの特性から、レンヌの中心から離れた場所を通過する光は近くを通過する光よりも大きく曲げられる。
この特性より、光学の中心において歪みは観測されなく、周辺部分に行くに従って歪みは大きくなる。
現実では、この歪みは小さいものであり、光学中心からみた半径を\begin{math}{r}\end{math}とし、\begin{math}{r=0}\end{math}付近でのテイラー級数を求め、近似値として表せる。
\begin{eqnarray}
  x_{corrected} & = & x(1+k_1 r^2+k_2 r^4+k_3 r^6) \\
  y_{corrected} & = & y(1+k_1 r^2+k_2 r^4+k_3 r^6)
\end{eqnarray}
一般的には、第二の項を用いるが、魚眼カメラのように歪みの大きいカメラの場合は第三の項を使って近似する。

____________________________________________

この他にも、レンズの組み合わせによる校正方法が挙げられる。
