\section{レンズの歪曲収差の校正方法について}
レンズの歪みには、レンズの形状による半径方向歪みとカメラの組み立ての不具合による円周方向歪みがある。
これらの歪みを考慮してカメラの特性を得ることをカメラキャリブレーションと呼ぶ。
つまり、カメラキャリブレーションは歪曲収差を校正しているといえる。ここで、カメラキャリブレーションの方法について説明する。
%歪曲収差は半径方向歪みに分類され、これはレンズの形状に起因する。

まず初めに、半径方向歪みについて考える。
レンズの特性から、レンズの中心から離れた場所を通過する光は近くを通過する光よりも大きく曲げられる。
この特性より、光学の中心において歪みは観測されなく、周辺部分に行くに従って歪みは大きくなる。
現実では、この歪みは小さいものであり、光学中心からみた半径を\begin{math}{r}\end{math}とし、\begin{math}{r=0}\end{math}付近でのテイラー級数を求め、近似値として表せる。
\begin{eqnarray}
  x_{corrected} & = & x(1+k_1 r^2+k_2 r^4+k_3 r^6) \\
  y_{corrected} & = & y(1+k_1 r^2+k_2 r^4+k_3 r^6)
\end{eqnarray}\\
ここで、\begin{math}x_{corrected},y_{corrected}は補正の結果で得られる位置である。\end{math} \\
次に、円周方向歪みについて考える。円周方向歪みはレンズが画像平面に対して完全に平行に取り付けられていないという欠陥によって起こる。
この歪みは2つのパラメータによって表現することができる。
\begin{eqnarray}
  x_{corrected} & = & x + [ 2p_1xy + p_2 (r^2 + 2x^2) ] \\
  y_{corrected} & = & y + [ p_1(r^2+2y^2) + 2p_2xy]
\end{eqnarray}\\
以上の4つの式から、合計で5つの歪み係数(\begin{math}k_1,k_2,k_3,p_1,p_2\end{math})を求めることによって歪みの補正ができる。
カメラの内部パラメータを用いて、歪み係数を求めることができる。また、ここで使用される内部パラメータは初期推定時のパラメータである。\\
ここで、歪みのない位置を\begin{math}(x_{p},y_{p})\end{math},歪みによりずれた位置を\begin{math}(x_{d},y_{d})\end{math}とすると,歪み係数を求めるための方程式は以下のようになる。
\begin{eqnarray}
\label{eq9}
\left[ \begin{array}{l}
x_{p} \\
y_{p}
\end{array} \right]
=(1+k_1 r^2+k_2 r^4+k_3 r^6)
\left[\begin{array}{l}
x_{d} \\
y_{d}
\end{array} \right]
+
\left[ \begin{array}{l}
  2p_1x_dy_d + p_2 (r^2 + 2x_d^2)  \\
  p_1(r^2+2y_d^2) + 2p_2x_dy_d
\end{array} \right]
\end{eqnarray}\\
\eqref{eq9}の方程式はキャリブレーションにより複数集めることで解くことができ、歪み係数を求めることができる。その後、内部パラメータや外部パラメータは再推定される。
つまり、レンズの校正方法とはカメラキャリブレーションという内部パラメータと外部パラメータを推定する過程の内でできる。
カメラキャリブレーションの一例として、チェスボードを複数枚とった方法が挙げられる。\\
また、カメラキャリブレーションによる方法以外にも、レンズの組み合わせや絞りの調整などの物理的に歪みによる位置ずれを回避することも可能である。